% Options for packages loaded elsewhere
\PassOptionsToPackage{unicode}{hyperref}
\PassOptionsToPackage{hyphens}{url}
%
\documentclass[
]{article}
\usepackage{lmodern}
\usepackage{amssymb,amsmath}
\usepackage{ifxetex,ifluatex}
\ifnum 0\ifxetex 1\fi\ifluatex 1\fi=0 % if pdftex
  \usepackage[T1]{fontenc}
  \usepackage[utf8]{inputenc}
  \usepackage{textcomp} % provide euro and other symbols
\else % if luatex or xetex
  \usepackage{unicode-math}
  \defaultfontfeatures{Scale=MatchLowercase}
  \defaultfontfeatures[\rmfamily]{Ligatures=TeX,Scale=1}
\fi
% Use upquote if available, for straight quotes in verbatim environments
\IfFileExists{upquote.sty}{\usepackage{upquote}}{}
\IfFileExists{microtype.sty}{% use microtype if available
  \usepackage[]{microtype}
  \UseMicrotypeSet[protrusion]{basicmath} % disable protrusion for tt fonts
}{}
\makeatletter
\@ifundefined{KOMAClassName}{% if non-KOMA class
  \IfFileExists{parskip.sty}{%
    \usepackage{parskip}
  }{% else
    \setlength{\parindent}{0pt}
    \setlength{\parskip}{6pt plus 2pt minus 1pt}}
}{% if KOMA class
  \KOMAoptions{parskip=half}}
\makeatother
\usepackage{xcolor}
\IfFileExists{xurl.sty}{\usepackage{xurl}}{} % add URL line breaks if available
\IfFileExists{bookmark.sty}{\usepackage{bookmark}}{\usepackage{hyperref}}
\hypersetup{
  pdftitle={Project1-Kaylei-Nilson},
  pdfauthor={Kaylei Nilson-Pierce},
  hidelinks,
  pdfcreator={LaTeX via pandoc}}
\urlstyle{same} % disable monospaced font for URLs
\usepackage[margin=1in]{geometry}
\usepackage{color}
\usepackage{fancyvrb}
\newcommand{\VerbBar}{|}
\newcommand{\VERB}{\Verb[commandchars=\\\{\}]}
\DefineVerbatimEnvironment{Highlighting}{Verbatim}{commandchars=\\\{\}}
% Add ',fontsize=\small' for more characters per line
\usepackage{framed}
\definecolor{shadecolor}{RGB}{248,248,248}
\newenvironment{Shaded}{\begin{snugshade}}{\end{snugshade}}
\newcommand{\AlertTok}[1]{\textcolor[rgb]{0.94,0.16,0.16}{#1}}
\newcommand{\AnnotationTok}[1]{\textcolor[rgb]{0.56,0.35,0.01}{\textbf{\textit{#1}}}}
\newcommand{\AttributeTok}[1]{\textcolor[rgb]{0.77,0.63,0.00}{#1}}
\newcommand{\BaseNTok}[1]{\textcolor[rgb]{0.00,0.00,0.81}{#1}}
\newcommand{\BuiltInTok}[1]{#1}
\newcommand{\CharTok}[1]{\textcolor[rgb]{0.31,0.60,0.02}{#1}}
\newcommand{\CommentTok}[1]{\textcolor[rgb]{0.56,0.35,0.01}{\textit{#1}}}
\newcommand{\CommentVarTok}[1]{\textcolor[rgb]{0.56,0.35,0.01}{\textbf{\textit{#1}}}}
\newcommand{\ConstantTok}[1]{\textcolor[rgb]{0.00,0.00,0.00}{#1}}
\newcommand{\ControlFlowTok}[1]{\textcolor[rgb]{0.13,0.29,0.53}{\textbf{#1}}}
\newcommand{\DataTypeTok}[1]{\textcolor[rgb]{0.13,0.29,0.53}{#1}}
\newcommand{\DecValTok}[1]{\textcolor[rgb]{0.00,0.00,0.81}{#1}}
\newcommand{\DocumentationTok}[1]{\textcolor[rgb]{0.56,0.35,0.01}{\textbf{\textit{#1}}}}
\newcommand{\ErrorTok}[1]{\textcolor[rgb]{0.64,0.00,0.00}{\textbf{#1}}}
\newcommand{\ExtensionTok}[1]{#1}
\newcommand{\FloatTok}[1]{\textcolor[rgb]{0.00,0.00,0.81}{#1}}
\newcommand{\FunctionTok}[1]{\textcolor[rgb]{0.00,0.00,0.00}{#1}}
\newcommand{\ImportTok}[1]{#1}
\newcommand{\InformationTok}[1]{\textcolor[rgb]{0.56,0.35,0.01}{\textbf{\textit{#1}}}}
\newcommand{\KeywordTok}[1]{\textcolor[rgb]{0.13,0.29,0.53}{\textbf{#1}}}
\newcommand{\NormalTok}[1]{#1}
\newcommand{\OperatorTok}[1]{\textcolor[rgb]{0.81,0.36,0.00}{\textbf{#1}}}
\newcommand{\OtherTok}[1]{\textcolor[rgb]{0.56,0.35,0.01}{#1}}
\newcommand{\PreprocessorTok}[1]{\textcolor[rgb]{0.56,0.35,0.01}{\textit{#1}}}
\newcommand{\RegionMarkerTok}[1]{#1}
\newcommand{\SpecialCharTok}[1]{\textcolor[rgb]{0.00,0.00,0.00}{#1}}
\newcommand{\SpecialStringTok}[1]{\textcolor[rgb]{0.31,0.60,0.02}{#1}}
\newcommand{\StringTok}[1]{\textcolor[rgb]{0.31,0.60,0.02}{#1}}
\newcommand{\VariableTok}[1]{\textcolor[rgb]{0.00,0.00,0.00}{#1}}
\newcommand{\VerbatimStringTok}[1]{\textcolor[rgb]{0.31,0.60,0.02}{#1}}
\newcommand{\WarningTok}[1]{\textcolor[rgb]{0.56,0.35,0.01}{\textbf{\textit{#1}}}}
\usepackage{graphicx,grffile}
\makeatletter
\def\maxwidth{\ifdim\Gin@nat@width>\linewidth\linewidth\else\Gin@nat@width\fi}
\def\maxheight{\ifdim\Gin@nat@height>\textheight\textheight\else\Gin@nat@height\fi}
\makeatother
% Scale images if necessary, so that they will not overflow the page
% margins by default, and it is still possible to overwrite the defaults
% using explicit options in \includegraphics[width, height, ...]{}
\setkeys{Gin}{width=\maxwidth,height=\maxheight,keepaspectratio}
% Set default figure placement to htbp
\makeatletter
\def\fps@figure{htbp}
\makeatother
\setlength{\emergencystretch}{3em} % prevent overfull lines
\providecommand{\tightlist}{%
  \setlength{\itemsep}{0pt}\setlength{\parskip}{0pt}}
\setcounter{secnumdepth}{-\maxdimen} % remove section numbering

\title{Project1-Kaylei-Nilson}
\author{Kaylei Nilson-Pierce}
\date{10/1/2020}

\begin{document}
\maketitle

Video link: \url{https://youtu.be/UAF352c1VIU}

\hypertarget{import-data-only-first-16-columns}{%
\subsubsection{Import data: only first 16
columns}\label{import-data-only-first-16-columns}}

\begin{Shaded}
\begin{Highlighting}[]
\NormalTok{col_names <-}\StringTok{ }\KeywordTok{c}\NormalTok{(}\StringTok{"SID"}\NormalTok{, }\StringTok{"SEASON"}\NormalTok{, }\StringTok{"NUMBER"}\NormalTok{, }\StringTok{"BASIN"}\NormalTok{, }\StringTok{"SUBBASIN"}\NormalTok{, }\StringTok{"NAME"}\NormalTok{, }\StringTok{"ISO_TIME"}\NormalTok{, }\StringTok{"NATURE"}\NormalTok{, }\StringTok{"LAT"}\NormalTok{, }\StringTok{"LON"}\NormalTok{, }\StringTok{"WMO_WIND"}\NormalTok{, }\StringTok{"WMO_PRES"}\NormalTok{, }\StringTok{"WMO_AGENCY"}\NormalTok{, }\StringTok{"TRACK_TYPE"}\NormalTok{, }\StringTok{"DIST2LAND"}\NormalTok{, }\StringTok{"LANDFALL"}\NormalTok{)}

\NormalTok{col_types=}\KeywordTok{c}\NormalTok{(}\StringTok{'character'}\NormalTok{,}\StringTok{"integer"}\NormalTok{,}\StringTok{"integer"}\NormalTok{, }\StringTok{"character"}\NormalTok{, }\StringTok{"character"}\NormalTok{, }\StringTok{'character'}\NormalTok{, }\StringTok{'character'}\NormalTok{,}\StringTok{'character'}\NormalTok{, }\StringTok{'double'}\NormalTok{,}\StringTok{'double'}\NormalTok{,}\StringTok{'integer'}\NormalTok{,}\StringTok{"character"}\NormalTok{, }\StringTok{'character'}\NormalTok{,}\StringTok{'character'}\NormalTok{,}\StringTok{'integer'}\NormalTok{, }\StringTok{'integer'}\NormalTok{,}\KeywordTok{rep}\NormalTok{(}\StringTok{'NULL'}\NormalTok{, }\DecValTok{147}\NormalTok{))}

\NormalTok{dat <-}\StringTok{ }\KeywordTok{read.csv}\NormalTok{(}\DataTypeTok{file=}\StringTok{'ibtracs.NA.list.v04r00.csv'}\NormalTok{, }\DataTypeTok{skip=}\DecValTok{86272}\NormalTok{,}
         \DataTypeTok{colClass=}\NormalTok{col_types, }\DataTypeTok{stringsAsFactors =} \OtherTok{FALSE}\NormalTok{, }\DataTypeTok{na.strings =} \StringTok{"MM"}\NormalTok{)}

\KeywordTok{colnames}\NormalTok{(dat) <-}\StringTok{ }\NormalTok{col_names}

\KeywordTok{head}\NormalTok{(dat, }\DecValTok{5}\NormalTok{)}
\end{Highlighting}
\end{Shaded}

\begin{verbatim}
##             SID SEASON NUMBER BASIN SUBBASIN      NAME            ISO_TIME
## 1 1980199N31284   1980     49    NA       NA NOT_NAMED 1980-07-17 00:00:00
## 2 1980199N31284   1980     49    NA       NA NOT_NAMED 1980-07-17 03:00:00
## 3 1980199N31284   1980     49    NA       NA NOT_NAMED 1980-07-17 06:00:00
## 4 1980199N31284   1980     49    NA       NA NOT_NAMED 1980-07-17 09:00:00
## 5 1980199N31284   1980     49    NA       NA NOT_NAMED 1980-07-17 12:00:00
##   NATURE     LAT      LON WMO_WIND WMO_PRES WMO_AGENCY TRACK_TYPE DIST2LAND
## 1     TS 30.5000 -76.5000       20          hurdat_atl       main       390
## 2     TS 30.3428 -76.8528       NA                           main       382
## 3     TS 30.2000 -77.2000       25          hurdat_atl       main       371
## 4     TS 30.0845 -77.5549       NA                           main       332
## 5     TS 30.0000 -78.0000       25          hurdat_atl       main       294
##   LANDFALL
## 1      379
## 2      371
## 3      341
## 4      294
## 5      239
\end{verbatim}

\hypertarget{add-month-column}{%
\paragraph{Add month column}\label{add-month-column}}

\begin{Shaded}
\begin{Highlighting}[]
\NormalTok{dat}\OperatorTok{$}\NormalTok{MONTH <-}\KeywordTok{as.numeric}\NormalTok{(}\KeywordTok{substr}\NormalTok{(dat}\OperatorTok{$}\NormalTok{ISO_TIME, }\DecValTok{6}\NormalTok{, }\DecValTok{7}\NormalTok{))}
\KeywordTok{str}\NormalTok{(dat, }\DataTypeTok{vec.len =} \DecValTok{1}\NormalTok{)}
\end{Highlighting}
\end{Shaded}

\begin{verbatim}
## 'data.frame':    36069 obs. of  17 variables:
##  $ SID       : chr  "1980199N31284" ...
##  $ SEASON    : int  1980 1980 ...
##  $ NUMBER    : int  49 49 ...
##  $ BASIN     : chr  "NA" ...
##  $ SUBBASIN  : chr  "NA" ...
##  $ NAME      : chr  "NOT_NAMED" ...
##  $ ISO_TIME  : chr  "1980-07-17 00:00:00" ...
##  $ NATURE    : chr  "TS" ...
##  $ LAT       : num  30.5 ...
##  $ LON       : num  -76.5 ...
##  $ WMO_WIND  : int  20 NA ...
##  $ WMO_PRES  : chr  " " ...
##  $ WMO_AGENCY: chr  "hurdat_atl" ...
##  $ TRACK_TYPE: chr  "main" ...
##  $ DIST2LAND : int  390 382 ...
##  $ LANDFALL  : int  379 371 ...
##  $ MONTH     : num  7 7 ...
\end{verbatim}

\hypertarget{manipulating-data-frames-that-will-be-used-later-on.}{%
\subsubsection{Manipulating data frames that will be used later
on.}\label{manipulating-data-frames-that-will-be-used-later-on.}}

\hypertarget{we-will-only-be-exploring-data-in-19802019.}{%
\paragraph{We will only be exploring data in
1980:2019.}\label{we-will-only-be-exploring-data-in-19802019.}}

\begin{Shaded}
\begin{Highlighting}[]
\NormalTok{dat2 <-}\StringTok{ }\KeywordTok{filter}\NormalTok{(dat, SEASON }\OperatorTok\StringTok{ }\DecValTok{1980}\OperatorTok{:}\DecValTok{2019}\NormalTok{)}
\KeywordTok{head}\NormalTok{(dat2, }\DecValTok{5}\NormalTok{)}
\end{Highlighting}
\end{Shaded}

\begin{verbatim}
##             SID SEASON NUMBER BASIN SUBBASIN      NAME            ISO_TIME
## 1 1980199N31284   1980     49    NA       NA NOT_NAMED 1980-07-17 00:00:00
## 2 1980199N31284   1980     49    NA       NA NOT_NAMED 1980-07-17 03:00:00
## 3 1980199N31284   1980     49    NA       NA NOT_NAMED 1980-07-17 06:00:00
## 4 1980199N31284   1980     49    NA       NA NOT_NAMED 1980-07-17 09:00:00
## 5 1980199N31284   1980     49    NA       NA NOT_NAMED 1980-07-17 12:00:00
##   NATURE     LAT      LON WMO_WIND WMO_PRES WMO_AGENCY TRACK_TYPE DIST2LAND
## 1     TS 30.5000 -76.5000       20          hurdat_atl       main       390
## 2     TS 30.3428 -76.8528       NA                           main       382
## 3     TS 30.2000 -77.2000       25          hurdat_atl       main       371
## 4     TS 30.0845 -77.5549       NA                           main       332
## 5     TS 30.0000 -78.0000       25          hurdat_atl       main       294
##   LANDFALL MONTH
## 1      379     7
## 2      371     7
## 3      341     7
## 4      294     7
## 5      239     7
\end{verbatim}

\hypertarget{create-column-hurricane}{%
\paragraph{Create column ``HURRICANE''}\label{create-column-hurricane}}

The WMO\_WIND column provides its numbers in knots. According to the
textbook and websites provided, when a storm's sustained wind speed
reaches 74 mph or 64 kt it is considered a hurricane. Based off this
knowledge, I created a column that returned T/F values if the recorded
storm's WMO\_WIND was recorded to be = or \textgreater{} than 64 kt.

\begin{Shaded}
\begin{Highlighting}[]
\NormalTok{hurricane <-}\StringTok{ }\KeywordTok{mutate}\NormalTok{(dat2, }\DataTypeTok{HURRICANE =}\NormalTok{ WMO_WIND }\OperatorTok{>=}\StringTok{ }\DecValTok{64}\NormalTok{)}
\KeywordTok{head}\NormalTok{(hurricane,}\DecValTok{5}\NormalTok{)}
\end{Highlighting}
\end{Shaded}

\begin{verbatim}
##             SID SEASON NUMBER BASIN SUBBASIN      NAME            ISO_TIME
## 1 1980199N31284   1980     49    NA       NA NOT_NAMED 1980-07-17 00:00:00
## 2 1980199N31284   1980     49    NA       NA NOT_NAMED 1980-07-17 03:00:00
## 3 1980199N31284   1980     49    NA       NA NOT_NAMED 1980-07-17 06:00:00
## 4 1980199N31284   1980     49    NA       NA NOT_NAMED 1980-07-17 09:00:00
## 5 1980199N31284   1980     49    NA       NA NOT_NAMED 1980-07-17 12:00:00
##   NATURE     LAT      LON WMO_WIND WMO_PRES WMO_AGENCY TRACK_TYPE DIST2LAND
## 1     TS 30.5000 -76.5000       20          hurdat_atl       main       390
## 2     TS 30.3428 -76.8528       NA                           main       382
## 3     TS 30.2000 -77.2000       25          hurdat_atl       main       371
## 4     TS 30.0845 -77.5549       NA                           main       332
## 5     TS 30.0000 -78.0000       25          hurdat_atl       main       294
##   LANDFALL MONTH HURRICANE
## 1      379     7     FALSE
## 2      371     7        NA
## 3      341     7     FALSE
## 4      294     7        NA
## 5      239     7     FALSE
\end{verbatim}

\hypertarget{create-column-to-identify-the-category-of-storms}{%
\paragraph{Create column to identify the category of
storms}\label{create-column-to-identify-the-category-of-storms}}

I used the function cut() to assign category values 0:5 to corresponding
WMO\_WIND values.

I refrenced the textbook, chapter 8, for the WMO\_WIND cutoffs that I
used to distinguish the different categories.

\begin{itemize}
\tightlist
\item
  category 1: 64-82 kt
\item
  category 2: 83-95 kt
\item
  category 3: 96-112 kt
\item
  category 4: 113-136 kt
\item
  category 5: 137 kt or higher
\end{itemize}

\begin{Shaded}
\begin{Highlighting}[]
\NormalTok{wind <-}\StringTok{ }\NormalTok{hurricane}\OperatorTok{$}\NormalTok{WMO_WIND}
\NormalTok{category <-}\StringTok{ }\KeywordTok{cut}\NormalTok{(wind, }\DataTypeTok{breaks=} \KeywordTok{c}\NormalTok{(}\DecValTok{0}\NormalTok{, }\DecValTok{63}\NormalTok{, }\DecValTok{82}\NormalTok{, }\DecValTok{95}\NormalTok{, }\DecValTok{112}\NormalTok{, }\DecValTok{136}\NormalTok{, }\OtherTok{Inf}\NormalTok{), }\DataTypeTok{labels=} \KeywordTok{c}\NormalTok{(}\StringTok{"0"}\NormalTok{,}\StringTok{"1"}\NormalTok{,}\StringTok{"2"}\NormalTok{,}\StringTok{"3"}\NormalTok{,}\StringTok{"4"}\NormalTok{,}\StringTok{"5"}\NormalTok{))}
\end{Highlighting}
\end{Shaded}

\begin{Shaded}
\begin{Highlighting}[]
\NormalTok{dat3 <-}\StringTok{ }\KeywordTok{mutate}\NormalTok{(hurricane, }\DataTypeTok{CATEGORY=}\NormalTok{ category)}
\KeywordTok{head}\NormalTok{(dat3, }\DecValTok{5}\NormalTok{)}
\end{Highlighting}
\end{Shaded}

\begin{verbatim}
##             SID SEASON NUMBER BASIN SUBBASIN      NAME            ISO_TIME
## 1 1980199N31284   1980     49    NA       NA NOT_NAMED 1980-07-17 00:00:00
## 2 1980199N31284   1980     49    NA       NA NOT_NAMED 1980-07-17 03:00:00
## 3 1980199N31284   1980     49    NA       NA NOT_NAMED 1980-07-17 06:00:00
## 4 1980199N31284   1980     49    NA       NA NOT_NAMED 1980-07-17 09:00:00
## 5 1980199N31284   1980     49    NA       NA NOT_NAMED 1980-07-17 12:00:00
##   NATURE     LAT      LON WMO_WIND WMO_PRES WMO_AGENCY TRACK_TYPE DIST2LAND
## 1     TS 30.5000 -76.5000       20          hurdat_atl       main       390
## 2     TS 30.3428 -76.8528       NA                           main       382
## 3     TS 30.2000 -77.2000       25          hurdat_atl       main       371
## 4     TS 30.0845 -77.5549       NA                           main       332
## 5     TS 30.0000 -78.0000       25          hurdat_atl       main       294
##   LANDFALL MONTH HURRICANE CATEGORY
## 1      379     7     FALSE        0
## 2      371     7        NA     <NA>
## 3      341     7     FALSE        0
## 4      294     7        NA     <NA>
## 5      239     7     FALSE        0
\end{verbatim}

\hypertarget{exploratory-analysis}{%
\subsection{Exploratory analysis}\label{exploratory-analysis}}

\hypertarget{basin-and-subbasin}{%
\subsubsection{BASIN and SUBBASIN}\label{basin-and-subbasin}}

I have been asked to analyze hurricane data in the North Atlantic. When
looking through the values in BASIN both NA and EP (Eastern Pacific) are
logged. This lead me to have two theories:

\begin{enumerate}
\def\labelenumi{\arabic{enumi})}
\tightlist
\item
  the EP values were misrecorded since there's only a small minority of
  them and they are actually NA
\item
  the EP values do correspond to the Eastern Pacific and the data I am
  working with is not limited to the North Alantic as I had initially
  assumed.
\end{enumerate}

In order to answer my question, I plotted the data set by latitude and
longitude and colored by BASIN (NA and EP values) to see where they show
up.

\begin{Shaded}
\begin{Highlighting}[]
\NormalTok{world_map <-}\StringTok{ }\KeywordTok{map_data}\NormalTok{(}\StringTok{"world"}\NormalTok{)}

\NormalTok{gg_world <-}\StringTok{ }\KeywordTok{ggplot}\NormalTok{() }\OperatorTok{+}\StringTok{ }
\StringTok{  }\KeywordTok{geom_polygon}\NormalTok{(}\DataTypeTok{data =}\NormalTok{ world_map, }
               \KeywordTok{aes}\NormalTok{(}\DataTypeTok{x =}\NormalTok{ long, }\DataTypeTok{y =}\NormalTok{ lat, }\DataTypeTok{group =}\NormalTok{ group), }
               \DataTypeTok{fill =} \StringTok{"gray95"}\NormalTok{, }\DataTypeTok{colour =} \StringTok{"gray70"}\NormalTok{, }\DataTypeTok{size =} \FloatTok{0.2}\NormalTok{) }\OperatorTok{+}
\StringTok{  }\KeywordTok{theme_bw}\NormalTok{()}
\end{Highlighting}
\end{Shaded}

\begin{Shaded}
\begin{Highlighting}[]
\NormalTok{gg_world }\OperatorTok{+}\StringTok{ }
\StringTok{  }\KeywordTok{geom_point}\NormalTok{(}\DataTypeTok{data =}\NormalTok{ dat3, }\KeywordTok{aes}\NormalTok{(}\DataTypeTok{x =}\NormalTok{ LON, }\DataTypeTok{y =}\NormalTok{ LAT, }\DataTypeTok{color =}\NormalTok{ BASIN))}
\end{Highlighting}
\end{Shaded}

\includegraphics{KNFinalProject1_files/figure-latex/unnamed-chunk-8-1.pdf}

When I plotted the NA and EP values, they showed up in two differing
areas and the EP values did in fact correspond to the Eastern Pacific.

After consulting my professor, I was told it was best to use all the
data and not exclude EP values. Thus, the majority of the data points
are from the Noth Atlantic, but some of the data points are from the
Eastern Pacific.

\hypertarget{subbasin}{%
\subsection{SUBBASIN}\label{subbasin}}

Another discrepency I notice is that SUBBASIN includes NA values, but
the data dictionary says it should only include the following labels:
MM, CS, GM, CP, BB, AS, WA, and EA. I have two theories:

\begin{enumerate}
\def\labelenumi{\arabic{enumi})}
\tightlist
\item
  the first is that the NA values in SUBBASIN mean it is located in the
  North Atlantic and NA was accidentally used to describe SUBBASIN
  column as it is used to describe in the BASIN column
\item
  the NA values stand for values that are ``not applicable.'' In other
  words, the NA values do not represent the North Atlantic.
\end{enumerate}

Again, I plotted the latitude and longitude and colored by SUBBASIN to
see the results.

\begin{Shaded}
\begin{Highlighting}[]
\NormalTok{gg_world }\OperatorTok{+}\StringTok{ }
\StringTok{  }\KeywordTok{geom_point}\NormalTok{(}\DataTypeTok{data =}\NormalTok{ dat3, }\KeywordTok{aes}\NormalTok{(}\DataTypeTok{x =}\NormalTok{ LON, }\DataTypeTok{y =}\NormalTok{ LAT, }\DataTypeTok{color =}\NormalTok{ SUBBASIN))}
\end{Highlighting}
\end{Shaded}

\includegraphics{KNFinalProject1_files/figure-latex/unnamed-chunk-9-1.pdf}

The results of this map were supprising to me because they actually
confirm that both of my theories are correct. The majority of the NA
values really do correspond to the North Atlantic and were probably
mistankingly used in the SUBBASIN column instead of the BASIN column.
Notice that another group of NA values are gray and do not correspond to
the North Atlantic. Interestingly, notice that the gray NA values
correspond to the Eastern Pacific values (EP).

It would be difficult to filter out the gray NA values from the
legitamite blue NA values in the SUBBASIN category since you can't
really say exclude ``NA.'' However, since the gray NA values are located
in the EP, we can filter BASIN to exclude EP values then map again
according to SUBBASIN and see if this got rid of the gray NA values for
SUBBASIN.

\begin{Shaded}
\begin{Highlighting}[]
\NormalTok{NAONLY <-}\StringTok{ }\KeywordTok{filter}\NormalTok{(dat3, BASIN }\OperatorTok{!=}\StringTok{ "EP"}\NormalTok{)}

\NormalTok{gg_world }\OperatorTok{+}\StringTok{ }
\StringTok{  }\KeywordTok{geom_point}\NormalTok{(}\DataTypeTok{data =}\NormalTok{ NAONLY, }\KeywordTok{aes}\NormalTok{(}\DataTypeTok{x =}\NormalTok{ LON, }\DataTypeTok{y =}\NormalTok{ LAT, }\DataTypeTok{color =}\NormalTok{ SUBBASIN))}
\end{Highlighting}
\end{Shaded}

\includegraphics{KNFinalProject1_files/figure-latex/unnamed-chunk-10-1.pdf}

This map is accurate to the locations it is named as. If this were a
legitimate report that depended on absolute accuracy, we would probably
use the more filtered data set like this without the EP values.

\hypertarget{season-wmo_wind-and-category}{%
\subsection{SEASON, WMO\_WIND, and
CATEGORY}\label{season-wmo_wind-and-category}}

I am curious to see the distribution of WMO\_WIND and CATEGORY over time
to see if there is any pattern.

\begin{Shaded}
\begin{Highlighting}[]
\NormalTok{WindVals <-}\StringTok{ }\KeywordTok{filter}\NormalTok{(dat3, WMO_WIND }\OperatorTok{!=}\StringTok{ "is.na"}\NormalTok{)}

\KeywordTok{ggplot}\NormalTok{(WindVals, }\KeywordTok{aes}\NormalTok{(}\DataTypeTok{x=}\KeywordTok{factor}\NormalTok{(SEASON), }\DataTypeTok{y=}\NormalTok{WMO_WIND)) }\OperatorTok{+}
\StringTok{  }\KeywordTok{geom_point}\NormalTok{(}\KeywordTok{aes}\NormalTok{(}\DataTypeTok{color=}\NormalTok{CATEGORY))}\OperatorTok{+}\StringTok{ }
\StringTok{  }\KeywordTok{labs}\NormalTok{(}\DataTypeTok{x=} \StringTok{"Year"}\NormalTok{, }\DataTypeTok{title=} \StringTok{"Categories in Correlation to WMO_WIND"}\NormalTok{) }\OperatorTok{+}
\StringTok{  }\KeywordTok{theme}\NormalTok{(}\DataTypeTok{axis.text.x =} \KeywordTok{element_text}\NormalTok{(}\DataTypeTok{angle =} \DecValTok{90}\NormalTok{)) }
\end{Highlighting}
\end{Shaded}

\includegraphics{KNFinalProject1_files/figure-latex/unnamed-chunk-11-1.pdf}

\hypertarget{season-wmo_wind-and-nature}{%
\subsection{SEASON, WMO\_WIND and
NATURE}\label{season-wmo_wind-and-nature}}

After visualizing the correlation between WMO\_WIND and CATEGORY, I was
also curious to explore the NATURE of the storms and any correlation to
WMO\_WIND.

\begin{Shaded}
\begin{Highlighting}[]
\KeywordTok{ggplot}\NormalTok{(WindVals, }\KeywordTok{aes}\NormalTok{(}\DataTypeTok{x=}\KeywordTok{factor}\NormalTok{(SEASON), }\DataTypeTok{y=}\NormalTok{WMO_WIND)) }\OperatorTok{+}
\StringTok{  }\KeywordTok{geom_point}\NormalTok{(}\KeywordTok{aes}\NormalTok{(}\DataTypeTok{color=}\NormalTok{NATURE))}\OperatorTok{+}\StringTok{ }\KeywordTok{labs}\NormalTok{(}\DataTypeTok{x=} \StringTok{"Year"}\NormalTok{, }\DataTypeTok{title=} \StringTok{"Nature of Storms in Correlation to WMO_WIND"}\NormalTok{) }\OperatorTok{+}
\StringTok{  }\KeywordTok{theme}\NormalTok{(}\DataTypeTok{axis.text.x =} \KeywordTok{element_text}\NormalTok{(}\DataTypeTok{angle =} \DecValTok{90}\NormalTok{)) }
\end{Highlighting}
\end{Shaded}

\includegraphics{KNFinalProject1_files/figure-latex/unnamed-chunk-12-1.pdf}

\hypertarget{report}{%
\section{REPORT}\label{report}}

\hypertarget{claim-a-partial}{%
\section{Claim A: Partial}\label{claim-a-partial}}

\hypertarget{claim-a-a-typical-hurricane-season-during-a-calendar-year-runs-from-june-through-november-but-occasionally-storms-form-outside-those-months.}{%
\subsection{Claim A) A typical hurricane season (during a calendar year)
runs from June through November, but occasionally storms form outside
those
months.}\label{claim-a-a-typical-hurricane-season-during-a-calendar-year-runs-from-june-through-november-but-occasionally-storms-form-outside-those-months.}}

This claim is a bit ambiguous in that it could be interpreted in two
ways:

\begin{enumerate}
\def\labelenumi{\arabic{enumi})}
\tightlist
\item
  We look at hurricanes to answer the claim since the beginning of the
  statement says ``a typical hurricane season''
\item
  We could look at storms because the second part of the claim states
  ``occasionally storms form outside those months''
\end{enumerate}

This contradiction of using a hurricane season, but also showing storms
for the outside months led me to come up with my own solution. I decided
I will plot storms and hurricanes to show the difference in distribution
between the two so we can answer both parts of the claim.

First I will show a general visualization. I will seperate the storms
and hurricanes, group by month, and calculate the mean number of
occurances for each month over the time period 1980:2019.

\begin{Shaded}
\begin{Highlighting}[]
\NormalTok{hurricane.na <-}\StringTok{ }\KeywordTok{filter}\NormalTok{(dat3, HURRICANE }\OperatorTok{!=}\StringTok{ "is.na"}\NormalTok{)}

\NormalTok{storms_hurricanes <-}\StringTok{ }\KeywordTok{summarise}\NormalTok{(}
  \KeywordTok{group_by}\NormalTok{(hurricane.na, HURRICANE, SEASON, MONTH),}
  \DataTypeTok{number =} \KeywordTok{n}\NormalTok{()}
\NormalTok{)}
\end{Highlighting}
\end{Shaded}

\begin{verbatim}
## `summarise()` regrouping output by 'HURRICANE', 'SEASON' (override with `.groups` argument)
\end{verbatim}

\begin{Shaded}
\begin{Highlighting}[]
\KeywordTok{head}\NormalTok{(storms_hurricanes,}\DecValTok{5}\NormalTok{)}
\end{Highlighting}
\end{Shaded}

\begin{verbatim}
## # A tibble: 5 x 4
## # Groups:   HURRICANE, SEASON [1]
##   HURRICANE SEASON MONTH number
##   <lgl>      <int> <dbl>  <int>
## 1 FALSE       1980     7     51
## 2 FALSE       1980     8     72
## 3 FALSE       1980     9     93
## 4 FALSE       1980    10     30
## 5 FALSE       1980    11     57
\end{verbatim}

\begin{Shaded}
\begin{Highlighting}[]
\NormalTok{mean_all <-}\StringTok{ }\KeywordTok{summarise}\NormalTok{(}
  \KeywordTok{group_by}\NormalTok{(storms_hurricanes, HURRICANE, MONTH),}
  \DataTypeTok{mean=} \KeywordTok{mean}\NormalTok{(number))}
\end{Highlighting}
\end{Shaded}

\begin{verbatim}
## `summarise()` regrouping output by 'HURRICANE' (override with `.groups` argument)
\end{verbatim}

\begin{Shaded}
\begin{Highlighting}[]
\KeywordTok{head}\NormalTok{(mean_all,}\DecValTok{10}\NormalTok{)}
\end{Highlighting}
\end{Shaded}

\begin{verbatim}
## # A tibble: 10 x 3
## # Groups:   HURRICANE [1]
##    HURRICANE MONTH  mean
##    <lgl>     <dbl> <dbl>
##  1 FALSE         1  30.5
##  2 FALSE         4  23.8
##  3 FALSE         5  16.2
##  4 FALSE         6  23.2
##  5 FALSE         7  40.1
##  6 FALSE         8  83.6
##  7 FALSE         9 116. 
##  8 FALSE        10  62.1
##  9 FALSE        11  29.1
## 10 FALSE        12  20.5
\end{verbatim}

\begin{Shaded}
\begin{Highlighting}[]
\KeywordTok{ggplot}\NormalTok{(mean_all, }\KeywordTok{aes}\NormalTok{(}\DataTypeTok{x=} \KeywordTok{factor}\NormalTok{(MONTH), }\DataTypeTok{y=}\NormalTok{mean, }\DataTypeTok{fill=}\NormalTok{ HURRICANE)) }\OperatorTok{+}\StringTok{ }
\StringTok{  }\KeywordTok{geom_bar}\NormalTok{(}\DataTypeTok{stat =} \StringTok{"identity"}\NormalTok{)}\OperatorTok{+}
\StringTok{  }\KeywordTok{labs}\NormalTok{(}\DataTypeTok{x=}\StringTok{"Month"}\NormalTok{, }\DataTypeTok{y=} \StringTok{"Average Frequency"}\NormalTok{, }\DataTypeTok{title=} \StringTok{"Average Storm and Hurricane Frequency Each Month (1980-2019)"}\NormalTok{)}
\end{Highlighting}
\end{Shaded}

\includegraphics{KNFinalProject1_files/figure-latex/unnamed-chunk-15-1.pdf}

This visualization shows the average (mean) distribution of hurricanes
(blue) and storms(orange) for each month over the time period 1980:2019.
This shows that there is a high frequency of hurricanes between months
7-12. This visualization also confirms that storms do occur on
``outside'' months.

From this visualization, we see that hurricane frequency is highest in
the months 7-10. Interestingly, we don't see very high values for month
6 or 11 which are supposed to be high according to claim A's typical
hurricane season assessment. It is interesting to note month 1 has a
higher frequency average than month 11 which is unexpected since we are
considering month 1 an outside month.

This graph gives us a good general overview of the difference in
distribution for storms and hurricanes. This graph also provides us with
questions to look farther into for instance does it appear to be a high
number of occurances for storms and hurricanes in month 1 because it is
skewed?

To look a little more into this, I'm going to plot a boxplot of the
hurricanes and storms so we can see their outliers.

\hypertarget{hurricane-box-plot}{%
\subsubsection{Hurricane Box Plot}\label{hurricane-box-plot}}

\begin{Shaded}
\begin{Highlighting}[]
\NormalTok{hurricanedat <-}\StringTok{ }\KeywordTok{filter}\NormalTok{(dat3, HURRICANE }\OperatorTok{==}\StringTok{ "TRUE"}\NormalTok{)}

\NormalTok{hurricanes_per_month <-}\StringTok{ }\KeywordTok{summarise}\NormalTok{(}
  \KeywordTok{group_by}\NormalTok{(hurricanedat, SEASON, MONTH),}
  \DataTypeTok{number =} \KeywordTok{n}\NormalTok{()}
\NormalTok{)}
\end{Highlighting}
\end{Shaded}

\begin{verbatim}
## `summarise()` regrouping output by 'SEASON' (override with `.groups` argument)
\end{verbatim}

\begin{Shaded}
\begin{Highlighting}[]
\KeywordTok{ggplot}\NormalTok{(hurricanes_per_month, }\KeywordTok{aes}\NormalTok{(}\DataTypeTok{x=} \KeywordTok{factor}\NormalTok{(MONTH), }\DataTypeTok{y=}\NormalTok{number)) }\OperatorTok{+}
\StringTok{  }\KeywordTok{geom_boxplot}\NormalTok{(}\DataTypeTok{color=} \StringTok{"black"}\NormalTok{, }\DataTypeTok{fill=} \StringTok{"lightblue"}\NormalTok{, }\DataTypeTok{width=} \FloatTok{0.9}\NormalTok{) }\OperatorTok{+}\StringTok{ }
\StringTok{  }\KeywordTok{labs}\NormalTok{(}\DataTypeTok{x=} \StringTok{"Month"}\NormalTok{, }\DataTypeTok{y=}\StringTok{"Frequency"}\NormalTok{, }\DataTypeTok{title=} \StringTok{"Number of Hurricanes that Occur Each Month Between 1980-2019"}\NormalTok{)}
\end{Highlighting}
\end{Shaded}

\includegraphics{KNFinalProject1_files/figure-latex/unnamed-chunk-16-1.pdf}

\hypertarget{analysis}{%
\subsubsection{Analysis}\label{analysis}}

This boxplot shows that month 7, 8, 9, and 10 have outliers that make
them right-skewed resulting in a higher mean. Despite this, months 7-12
do have higher mean occurances than months 1-6. This would support a
``typical hurricane season,'' to be months 7-12 and ``outside months,''
to be 1-6 in contrast to claim A's specified months 6-11 and 12-5
(respectively).

\hypertarget{storm-box-plot}{%
\subsubsection{Storm Box Plot}\label{storm-box-plot}}

\begin{Shaded}
\begin{Highlighting}[]
\NormalTok{stormdat <-}\StringTok{ }\KeywordTok{filter}\NormalTok{(dat3, HURRICANE }\OperatorTok{==}\StringTok{ "FALSE"}\NormalTok{)}

\NormalTok{storms_per_month <-}\StringTok{ }\KeywordTok{summarise}\NormalTok{(}
  \KeywordTok{group_by}\NormalTok{(stormdat, SEASON, MONTH),}
  \DataTypeTok{number =} \KeywordTok{n}\NormalTok{()}
\NormalTok{)}
\end{Highlighting}
\end{Shaded}

\begin{verbatim}
## `summarise()` regrouping output by 'SEASON' (override with `.groups` argument)
\end{verbatim}

\begin{Shaded}
\begin{Highlighting}[]
\KeywordTok{ggplot}\NormalTok{(storms_per_month, }\KeywordTok{aes}\NormalTok{(}\DataTypeTok{x=} \KeywordTok{factor}\NormalTok{(MONTH), }\DataTypeTok{y=}\NormalTok{number)) }\OperatorTok{+}
\StringTok{  }\KeywordTok{geom_boxplot}\NormalTok{(}\DataTypeTok{color=} \StringTok{"black"}\NormalTok{, }\DataTypeTok{fill=} \StringTok{"orange"}\NormalTok{, }\DataTypeTok{width=} \FloatTok{0.9}\NormalTok{) }\OperatorTok{+}\StringTok{ }
\StringTok{  }\KeywordTok{labs}\NormalTok{(}\DataTypeTok{x=} \StringTok{"Month"}\NormalTok{, }\DataTypeTok{y=}\StringTok{"Frequency"}\NormalTok{, }\DataTypeTok{title=} \StringTok{"Number of Storms that Occur Each Month Between 1980-2019"}\NormalTok{)}
\end{Highlighting}
\end{Shaded}

\includegraphics{KNFinalProject1_files/figure-latex/unnamed-chunk-17-1.pdf}

\hypertarget{analysis-1}{%
\subsubsection{Analysis}\label{analysis-1}}

Contrary to the idea I previously mentioned, month 1 for storms does not
appear to have an outlier. On the other hand, month 11 does have an
outier which makes it right-skewed, having a higher mean than is
actually accurate of the average. Looking at the boxplot, one can see
that month 11's mean without outliers is below that of month 1.

This discrepency leads me to think that ``occasionally storms form in
outside these months,'' could be wrong because the mean of month 1 is
higher than month 6 and 11. However, I want to look deeper into this
before I jump to conclusions, so I will visualize the distribution of
months for each year.

\begin{Shaded}
\begin{Highlighting}[]
\NormalTok{sh_decade1 <-}\StringTok{ }\KeywordTok{filter}\NormalTok{(storms_hurricanes, SEASON }\OperatorTok\StringTok{ }\DecValTok{1980}\OperatorTok{:}\DecValTok{1989}\NormalTok{)}

\KeywordTok{ggplot}\NormalTok{(sh_decade1, }\KeywordTok{aes}\NormalTok{(}\DataTypeTok{x=} \KeywordTok{factor}\NormalTok{(SEASON), }\DataTypeTok{y=}\NormalTok{number, }\DataTypeTok{fill=}\NormalTok{HURRICANE)) }\OperatorTok{+}
\StringTok{  }\KeywordTok{geom_bar}\NormalTok{(}\DataTypeTok{stat=} \StringTok{"identity"}\NormalTok{) }\OperatorTok{+}
\StringTok{  }\KeywordTok{labs}\NormalTok{(}\DataTypeTok{x=} \StringTok{"Year"}\NormalTok{, }\DataTypeTok{y=}\StringTok{"Frequency"}\NormalTok{, }\DataTypeTok{title=} \StringTok{"Number of Storms and Hurricanes Each Month (1980-1989)"}\NormalTok{)}\OperatorTok{+}
\StringTok{  }\KeywordTok{facet_wrap}\NormalTok{(}\OperatorTok{~}\NormalTok{MONTH)}\OperatorTok{+}
\StringTok{  }\KeywordTok{theme}\NormalTok{(}\DataTypeTok{axis.text.x =} \KeywordTok{element_text}\NormalTok{(}\DataTypeTok{angle =} \DecValTok{90}\NormalTok{))   }
\end{Highlighting}
\end{Shaded}

\includegraphics{KNFinalProject1_files/figure-latex/unnamed-chunk-18-1.pdf}

\hypertarget{summary}{%
\paragraph{Summary:}\label{summary}}

Month 1 does not appear. Month 4, 5, and 12 have low frequency. Month 7
seems slightly ahead of month 6. Month 11 seems slightly ahead of month
7. Month 8,9,10 have the highest occurances.

\begin{Shaded}
\begin{Highlighting}[]
\NormalTok{sh_decade2 <-}\StringTok{ }\KeywordTok{filter}\NormalTok{(storms_hurricanes, SEASON }\OperatorTok\StringTok{ }\DecValTok{1990}\OperatorTok{:}\DecValTok{1999}\NormalTok{)}

\KeywordTok{ggplot}\NormalTok{(sh_decade2, }\KeywordTok{aes}\NormalTok{(}\DataTypeTok{x=} \KeywordTok{factor}\NormalTok{(SEASON), }\DataTypeTok{y=}\NormalTok{number, }\DataTypeTok{fill=}\NormalTok{HURRICANE)) }\OperatorTok{+}
\StringTok{  }\KeywordTok{geom_bar}\NormalTok{(}\DataTypeTok{stat=} \StringTok{"identity"}\NormalTok{) }\OperatorTok{+}
\StringTok{  }\KeywordTok{labs}\NormalTok{(}\DataTypeTok{x=} \StringTok{"Year"}\NormalTok{, }\DataTypeTok{y=}\StringTok{"Frequency"}\NormalTok{, }\DataTypeTok{title=} \StringTok{"Number of Storms and Hurricanes Each Month (1990-2000)"}\NormalTok{)}\OperatorTok{+}
\StringTok{  }\KeywordTok{facet_wrap}\NormalTok{(}\OperatorTok{~}\NormalTok{MONTH)}\OperatorTok{+}
\StringTok{  }\KeywordTok{theme}\NormalTok{(}\DataTypeTok{axis.text.x =} \KeywordTok{element_text}\NormalTok{(}\DataTypeTok{angle =} \DecValTok{90}\NormalTok{))   }
\end{Highlighting}
\end{Shaded}

\includegraphics{KNFinalProject1_files/figure-latex/unnamed-chunk-19-1.pdf}

\hypertarget{summary-1}{%
\paragraph{Summary:}\label{summary-1}}

Month 1 does not appear. Month 4, 5, and 12 have low frequency. Month 7
has higher frequency than month 6. Month 11 seems slightly less than
month 7. Month 8,9,10 have the highest occurances.

\begin{Shaded}
\begin{Highlighting}[]
\NormalTok{sh_decade3 <-}\StringTok{ }\KeywordTok{filter}\NormalTok{(storms_hurricanes, SEASON }\OperatorTok\StringTok{ }\DecValTok{2000}\OperatorTok{:}\DecValTok{2009}\NormalTok{)}

\KeywordTok{ggplot}\NormalTok{(sh_decade3, }\KeywordTok{aes}\NormalTok{(}\DataTypeTok{x=} \KeywordTok{factor}\NormalTok{(SEASON), }\DataTypeTok{y=}\NormalTok{number, }\DataTypeTok{fill=}\NormalTok{HURRICANE)) }\OperatorTok{+}
\StringTok{  }\KeywordTok{geom_bar}\NormalTok{(}\DataTypeTok{stat=} \StringTok{"identity"}\NormalTok{) }\OperatorTok{+}
\StringTok{  }\KeywordTok{labs}\NormalTok{(}\DataTypeTok{x=} \StringTok{"Year"}\NormalTok{, }\DataTypeTok{y=}\StringTok{"Frequency"}\NormalTok{, }\DataTypeTok{title=} \StringTok{"Number of Storms and Hurricanes Each Month (2000-2009)"}\NormalTok{)}\OperatorTok{+}
\StringTok{  }\KeywordTok{facet_wrap}\NormalTok{(}\OperatorTok{~}\NormalTok{MONTH)}\OperatorTok{+}
\StringTok{  }\KeywordTok{theme}\NormalTok{(}\DataTypeTok{axis.text.x =} \KeywordTok{element_text}\NormalTok{(}\DataTypeTok{angle =} \DecValTok{90}\NormalTok{))   }
\end{Highlighting}
\end{Shaded}

\includegraphics{KNFinalProject1_files/figure-latex/unnamed-chunk-20-1.pdf}

\hypertarget{summary-2}{%
\paragraph{Summary:}\label{summary-2}}

Month 1 appears. Month 4, 5, and 12 have low frequency. Month 7 has
higher frequency than month 6. Month 11 appears the same amount of times
as month 7 but has less frequency. Month 8,9,10 have the highest
occurances.

\begin{Shaded}
\begin{Highlighting}[]
\NormalTok{sh_decade4 <-}\StringTok{ }\KeywordTok{filter}\NormalTok{(storms_hurricanes, SEASON }\OperatorTok\StringTok{ }\DecValTok{2010}\OperatorTok{:}\DecValTok{2019}\NormalTok{)}

\KeywordTok{ggplot}\NormalTok{(sh_decade4, }\KeywordTok{aes}\NormalTok{(}\DataTypeTok{x=} \KeywordTok{factor}\NormalTok{(SEASON), }\DataTypeTok{y=}\NormalTok{number, }\DataTypeTok{fill=}\NormalTok{HURRICANE)) }\OperatorTok{+}
\StringTok{  }\KeywordTok{geom_bar}\NormalTok{(}\DataTypeTok{stat=} \StringTok{"identity"}\NormalTok{) }\OperatorTok{+}
\StringTok{  }\KeywordTok{labs}\NormalTok{(}\DataTypeTok{x=} \StringTok{"Year"}\NormalTok{, }\DataTypeTok{y=}\StringTok{"Frequency"}\NormalTok{, }\DataTypeTok{title=} \StringTok{"Number of Storms and Hurricanes Each Month (2010-2019)"}\NormalTok{)}\OperatorTok{+}
\StringTok{  }\KeywordTok{facet_wrap}\NormalTok{(}\OperatorTok{~}\NormalTok{MONTH)}\OperatorTok{+}
\StringTok{  }\KeywordTok{theme}\NormalTok{(}\DataTypeTok{axis.text.x =} \KeywordTok{element_text}\NormalTok{(}\DataTypeTok{angle =} \DecValTok{90}\NormalTok{))   }
\end{Highlighting}
\end{Shaded}

\includegraphics{KNFinalProject1_files/figure-latex/unnamed-chunk-21-1.pdf}

\hypertarget{summary-3}{%
\paragraph{Summary:}\label{summary-3}}

Month 1 appears. Month 4, 5, and 12 have low frequency. Month 7 has
higher frequency than month 6. Month 11 appears about the same as month
6 but has less frequency. Month 8,9,10 have the highest occurances.

\begin{Shaded}
\begin{Highlighting}[]
\NormalTok{sh_decade4 <-}\StringTok{ }\KeywordTok{filter}\NormalTok{(storms_hurricanes, SEASON }\OperatorTok\StringTok{ }\DecValTok{2010}\OperatorTok{:}\DecValTok{2019}\NormalTok{)}

\KeywordTok{ggplot}\NormalTok{(sh_decade4, }\KeywordTok{aes}\NormalTok{(}\DataTypeTok{x=} \KeywordTok{factor}\NormalTok{(SEASON), }\DataTypeTok{y=}\NormalTok{number, }\DataTypeTok{fill=}\NormalTok{HURRICANE)) }\OperatorTok{+}
\StringTok{  }\KeywordTok{geom_bar}\NormalTok{(}\DataTypeTok{stat=} \StringTok{"identity"}\NormalTok{) }\OperatorTok{+}
\StringTok{  }\KeywordTok{labs}\NormalTok{(}\DataTypeTok{x=} \StringTok{"Year"}\NormalTok{, }\DataTypeTok{y=}\StringTok{"Frequency"}\NormalTok{, }\DataTypeTok{title=} \StringTok{"Number of Storms and Hurricanes Each Month (2010-2019)"}\NormalTok{)}\OperatorTok{+}
\StringTok{  }\KeywordTok{facet_wrap}\NormalTok{(}\OperatorTok{~}\NormalTok{MONTH)}\OperatorTok{+}
\StringTok{  }\KeywordTok{theme}\NormalTok{(}\DataTypeTok{axis.text.x =} \KeywordTok{element_text}\NormalTok{(}\DataTypeTok{angle =} \DecValTok{90}\NormalTok{))   }
\end{Highlighting}
\end{Shaded}

\includegraphics{KNFinalProject1_files/figure-latex/unnamed-chunk-22-1.pdf}

\hypertarget{summary-4}{%
\paragraph{Summary:}\label{summary-4}}

Month 1 appears. Month 4, 5, and 12 have low frequency. Month 7 has
higher frequency than month 6. Month 11 appears about the same as month
6 but has less frequency. Month 8,9,10 have the highest occurances.

\hypertarget{final-conclusion-on-claim-a}{%
\subsubsection{Final Conclusion on Claim
A:}\label{final-conclusion-on-claim-a}}

I believe that the highest occurance of hurricanes occurs in the months
7-11. I think that hurricanes show up in month 6 at about the same
frequency as outside months. I would deem months 12-5 outside months.

Storms do occur outside of the typical hurricane season. After looking
at the decade distributions, I would describe them as ``ocasionally,''
occuring.

\hypertarget{claim-b-partial}{%
\section{Claim B: Partial}\label{claim-b-partial}}

\hypertarget{b-a-typical-year-has-12-named-storms-including-six-hurricanes-of-which-three-become-major-hurricanes-category-3-4-and-5.}{%
\subsection{B) A typical year has 12 named storms, including six
hurricanes of which three become major hurricanes (category 3, 4, and
5).}\label{b-a-typical-year-has-12-named-storms-including-six-hurricanes-of-which-three-become-major-hurricanes-category-3-4-and-5.}}

First I will isolate all named storms from dat3 and exclude any that are
listed as ``Not\_Named.''

\begin{Shaded}
\begin{Highlighting}[]
\NormalTok{named_storms <-}\StringTok{ }\KeywordTok{filter}\NormalTok{(dat3, NAME }\OperatorTok{!=}\StringTok{ "NOT_NAMED"}\NormalTok{)}
\KeywordTok{head}\NormalTok{(named_storms,}\DecValTok{5}\NormalTok{)}
\end{Highlighting}
\end{Shaded}

\begin{verbatim}
##             SID SEASON NUMBER BASIN SUBBASIN  NAME            ISO_TIME NATURE
## 1 1980214N11330   1980     57    NA       NA ALLEN 1980-07-31 12:00:00     NR
## 2 1980214N11330   1980     57    NA       NA ALLEN 1980-07-31 15:00:00     NR
## 3 1980214N11330   1980     57    NA       NA ALLEN 1980-07-31 18:00:00     NR
## 4 1980214N11330   1980     57    NA       NA ALLEN 1980-07-31 21:00:00     NR
## 5 1980214N11330   1980     57    NA       NA ALLEN 1980-08-01 00:00:00     TS
##       LAT      LON WMO_WIND WMO_PRES WMO_AGENCY TRACK_TYPE DIST2LAND LANDFALL
## 1 11.0000 -30.0000       25          hurdat_atl       main      1417     1417
## 2 10.9509 -31.1101       NA                           main      1531     1531
## 3 10.9000 -32.2000       25          hurdat_atl       main      1650     1650
## 4 10.8496 -33.2574       NA                           main      1695     1655
## 5 10.8000 -34.3000       30     1010 hurdat_atl       main      1651     1603
##   MONTH HURRICANE CATEGORY
## 1     7     FALSE        0
## 2     7        NA     <NA>
## 3     7     FALSE        0
## 4     7        NA     <NA>
## 5     8     FALSE        0
\end{verbatim}

I will filter out hurricanes by saying hurricane = false in order to
make sure I am only working with storms.

\begin{Shaded}
\begin{Highlighting}[]
\NormalTok{onlystorms <-}\StringTok{ }\KeywordTok{filter}\NormalTok{(named_storms, HURRICANE }\OperatorTok{==}\StringTok{ "FALSE"}\NormalTok{)}
\NormalTok{distinct_storms <-}\StringTok{ }\KeywordTok{distinct}\NormalTok{(}\KeywordTok{select}\NormalTok{(onlystorms, SEASON, NAME))}
\KeywordTok{head}\NormalTok{(distinct_storms, }\DecValTok{5}\NormalTok{)}
\end{Highlighting}
\end{Shaded}

\begin{verbatim}
##   SEASON    NAME
## 1   1980   ALLEN
## 2   1980  BONNIE
## 3   1980 CHARLEY
## 4   1980 GEORGES
## 5   1980    EARL
\end{verbatim}

Using the distinct named storm data (distinct\_storms), I will try to
verify the first part of the claim ``a typical year has 12 named
storms.'' I will create a table that counts the number of named storms
that occur for each season (year).

Based off of the table storms\_count, which provides the number of
storms that occured in each year, I will calculate the average using
mean.

\begin{Shaded}
\begin{Highlighting}[]
\NormalTok{storm_count <-}\StringTok{ }\KeywordTok{count}\NormalTok{(distinct_storms, SEASON)}
\NormalTok{storm_count }\OperatorTok\StringTok{ }\KeywordTok{summarise}\NormalTok{(}\DataTypeTok{mean_storms_season =} \KeywordTok{mean}\NormalTok{(n, }\DataTypeTok{na.rm =} \OtherTok{TRUE}\NormalTok{))}
\end{Highlighting}
\end{Shaded}

\begin{verbatim}
##   mean_storms_season
## 1             12.425
\end{verbatim}

This mean provides us an average of the amount of storms that occured
for each year. This average verifies that about 12 named storms did
occur each season. I will also visualize the data to see the trend.

\begin{Shaded}
\begin{Highlighting}[]
\NormalTok{storms_per_season <-}\StringTok{ }\KeywordTok{summarise}\NormalTok{(}
  \KeywordTok{group_by}\NormalTok{(distinct_storms, SEASON),}
  \DataTypeTok{NAMED_STORMS =} \KeywordTok{n}\NormalTok{()}
\NormalTok{)}
\end{Highlighting}
\end{Shaded}

\begin{verbatim}
## `summarise()` ungrouping output (override with `.groups` argument)
\end{verbatim}

\begin{Shaded}
\begin{Highlighting}[]
\KeywordTok{ggplot}\NormalTok{(}\DataTypeTok{data=}\NormalTok{storms_per_season, }\KeywordTok{aes}\NormalTok{(}\DataTypeTok{x=}\KeywordTok{factor}\NormalTok{(SEASON), }\DataTypeTok{y=}\NormalTok{NAMED_STORMS)) }\OperatorTok{+}
\KeywordTok{geom_bar}\NormalTok{(}\DataTypeTok{stat=}\StringTok{"identity"}\NormalTok{, }\DataTypeTok{color=} \StringTok{"black"}\NormalTok{, }\DataTypeTok{fill=} \StringTok{"lightblue"}\NormalTok{) }\OperatorTok{+}\StringTok{ }
\StringTok{  }\KeywordTok{scale_x_discrete}\NormalTok{(}\StringTok{"Year"}\NormalTok{, }\DataTypeTok{labels =} \DecValTok{1980}\OperatorTok{:}\DecValTok{2019}\NormalTok{, }\DataTypeTok{limits=} \KeywordTok{factor}\NormalTok{(}\DecValTok{1980}\OperatorTok{:}\DecValTok{2019}\NormalTok{)) }\OperatorTok{+}\StringTok{  }
\StringTok{  }\KeywordTok{labs}\NormalTok{(}\DataTypeTok{x=} \StringTok{"Year"}\NormalTok{, }\DataTypeTok{y=} \StringTok{"Number of Storms"}\NormalTok{, }\DataTypeTok{title=} \StringTok{"Number of Named Storms per Year (1980-2019)"}\NormalTok{)}\OperatorTok{+}\StringTok{  }
\StringTok{  }\KeywordTok{geom_abline}\NormalTok{(}\DataTypeTok{slope=}\DecValTok{0}\NormalTok{, }\DataTypeTok{intercept=}\DecValTok{12}\NormalTok{,  }\DataTypeTok{col =} \StringTok{"black"}\NormalTok{, }\DataTypeTok{lwd=}\DecValTok{2}\NormalTok{) }\OperatorTok{+}\StringTok{ }\KeywordTok{theme}\NormalTok{(}\DataTypeTok{axis.text.x =} \KeywordTok{element_text}\NormalTok{(}\DataTypeTok{angle =} \DecValTok{90}\NormalTok{))   }
\end{Highlighting}
\end{Shaded}

\includegraphics{KNFinalProject1_files/figure-latex/unnamed-chunk-26-1.pdf}

Next I will move to answering the next part of the claim: a typical year
includes 6 hurricanes. For this, I will filter hurricane = true.

\begin{Shaded}
\begin{Highlighting}[]
\NormalTok{named_hurricanes <-}\StringTok{ }\KeywordTok{filter}\NormalTok{(named_storms, HURRICANE }\OperatorTok{==}\StringTok{ "TRUE"}\NormalTok{)}
\NormalTok{distinct_hurricanes <-}\StringTok{ }\KeywordTok{distinct}\NormalTok{(}\KeywordTok{select}\NormalTok{(named_hurricanes, SEASON, NAME))}
\KeywordTok{head}\NormalTok{(distinct_hurricanes,}\DecValTok{5}\NormalTok{)}
\end{Highlighting}
\end{Shaded}

\begin{verbatim}
##   SEASON    NAME
## 1   1980   ALLEN
## 2   1980  BONNIE
## 3   1980 CHARLEY
## 4   1980 GEORGES
## 5   1980    EARL
\end{verbatim}

\begin{Shaded}
\begin{Highlighting}[]
\NormalTok{hurricanes_per_season <-}\StringTok{ }\KeywordTok{summarise}\NormalTok{(}
  \KeywordTok{group_by}\NormalTok{(distinct_hurricanes, SEASON),}
  \DataTypeTok{NAMED_HURRICANES =} \KeywordTok{n}\NormalTok{()}
\NormalTok{)}
\end{Highlighting}
\end{Shaded}

\begin{verbatim}
## `summarise()` ungrouping output (override with `.groups` argument)
\end{verbatim}

\begin{Shaded}
\begin{Highlighting}[]
\KeywordTok{head}\NormalTok{(hurricanes_per_season, }\DecValTok{5}\NormalTok{)}
\end{Highlighting}
\end{Shaded}

\begin{verbatim}
## # A tibble: 5 x 2
##   SEASON NAMED_HURRICANES
##    <int>            <int>
## 1   1980                9
## 2   1981                7
## 3   1982                2
## 4   1983                3
## 5   1984                5
\end{verbatim}

\begin{Shaded}
\begin{Highlighting}[]
\NormalTok{hurricanes_per_season <-}\StringTok{ }\KeywordTok{count}\NormalTok{(distinct_hurricanes, SEASON)}
\NormalTok{hurricanes_per_season }\OperatorTok\StringTok{ }\KeywordTok{summarise}\NormalTok{(}\DataTypeTok{mean_hurricane_season=} \KeywordTok{mean}\NormalTok{(n, }\DataTypeTok{na.rm=}\OtherTok{TRUE}\NormalTok{))}
\end{Highlighting}
\end{Shaded}

\begin{verbatim}
##   mean_hurricane_season
## 1                  6.65
\end{verbatim}

This value is slightly above what I would consider to still be about 6.
I am going to round up on this value and say it is closer to 7.
Therefore, I will have to state that I believe the second part of the
claim is false.

\begin{Shaded}
\begin{Highlighting}[]
\NormalTok{hurricanes_per_season <-}\StringTok{ }\KeywordTok{summarise}\NormalTok{(}
  \KeywordTok{group_by}\NormalTok{(distinct_hurricanes, SEASON),}
  \DataTypeTok{NAMED_HURRICANES =} \KeywordTok{n}\NormalTok{()}
\NormalTok{)}
\end{Highlighting}
\end{Shaded}

\begin{verbatim}
## `summarise()` ungrouping output (override with `.groups` argument)
\end{verbatim}

\begin{Shaded}
\begin{Highlighting}[]
\KeywordTok{ggplot}\NormalTok{(}\DataTypeTok{data=}\NormalTok{hurricanes_per_season, }\KeywordTok{aes}\NormalTok{(}\DataTypeTok{x=}\KeywordTok{factor}\NormalTok{(SEASON), }\DataTypeTok{y=}\NormalTok{NAMED_HURRICANES)) }\OperatorTok{+}
\KeywordTok{geom_bar}\NormalTok{(}\DataTypeTok{stat=}\StringTok{"identity"}\NormalTok{, }\DataTypeTok{color=} \StringTok{"black"}\NormalTok{, }\DataTypeTok{fill=} \StringTok{"lightblue"}\NormalTok{) }\OperatorTok{+}\StringTok{ }
\StringTok{  }\KeywordTok{scale_x_discrete}\NormalTok{(}\StringTok{"Year"}\NormalTok{, }\DataTypeTok{labels =} \DecValTok{1980}\OperatorTok{:}\DecValTok{2019}\NormalTok{, }\DataTypeTok{limits=} \KeywordTok{factor}\NormalTok{(}\DecValTok{1980}\OperatorTok{:}\DecValTok{2019}\NormalTok{)) }\OperatorTok{+}\StringTok{ }
\StringTok{  }\KeywordTok{labs}\NormalTok{(}\DataTypeTok{x=} \StringTok{"Year"}\NormalTok{, }\DataTypeTok{y=} \StringTok{"Number of Hurricanes"}\NormalTok{, }\DataTypeTok{title=} \StringTok{"Number of Named Hurricanes per Year (1980-2019)"}\NormalTok{)}\OperatorTok{+}\StringTok{  }\KeywordTok{geom_abline}\NormalTok{(}\DataTypeTok{slope=}\DecValTok{0}\NormalTok{, }\DataTypeTok{intercept=}\DecValTok{6}\NormalTok{,  }\DataTypeTok{col =} \StringTok{"black"}\NormalTok{, }\DataTypeTok{lwd=}\DecValTok{2}\NormalTok{) }\OperatorTok{+}
\KeywordTok{theme}\NormalTok{(}\DataTypeTok{axis.text.x =} \KeywordTok{element_text}\NormalTok{(}\DataTypeTok{angle =} \DecValTok{90}\NormalTok{)) }
\end{Highlighting}
\end{Shaded}

\includegraphics{KNFinalProject1_files/figure-latex/unnamed-chunk-30-1.pdf}

Finally, we will answer the third part of the claim which is three of
the hurricanes become major hurricanes per year. Major hurricanes are
classified by category 3, 4, and 5 which I have filtered for.

\begin{Shaded}
\begin{Highlighting}[]
\NormalTok{named_major <-}\StringTok{ }\KeywordTok{filter}\NormalTok{(named_storms, CATEGORY }\OperatorTok\StringTok{ }\DecValTok{3}\OperatorTok{:}\DecValTok{5}\NormalTok{)}
\NormalTok{distinct_major <-}\StringTok{ }\KeywordTok{distinct}\NormalTok{(}\KeywordTok{select}\NormalTok{(named_major, SEASON, NAME))}
\KeywordTok{head}\NormalTok{(distinct_major,}\DecValTok{5}\NormalTok{)}
\end{Highlighting}
\end{Shaded}

\begin{verbatim}
##   SEASON    NAME
## 1   1980   ALLEN
## 2   1980 FRANCES
## 3   1981   FLOYD
## 4   1981  HARVEY
## 5   1981   IRENE
\end{verbatim}

\begin{Shaded}
\begin{Highlighting}[]
\NormalTok{major_per_season <-}\StringTok{ }\KeywordTok{summarise}\NormalTok{(}
  \KeywordTok{group_by}\NormalTok{(distinct_major, SEASON),}
  \DataTypeTok{NAMED_MAJOR_HURRICANES =} \KeywordTok{n}\NormalTok{()}
\NormalTok{)}
\end{Highlighting}
\end{Shaded}

\begin{verbatim}
## `summarise()` ungrouping output (override with `.groups` argument)
\end{verbatim}

\begin{Shaded}
\begin{Highlighting}[]
\KeywordTok{head}\NormalTok{(major_per_season, }\DecValTok{5}\NormalTok{)}
\end{Highlighting}
\end{Shaded}

\begin{verbatim}
## # A tibble: 5 x 2
##   SEASON NAMED_MAJOR_HURRICANES
##    <int>                  <int>
## 1   1980                      2
## 2   1981                      3
## 3   1982                      1
## 4   1983                      1
## 5   1984                      1
\end{verbatim}

\begin{Shaded}
\begin{Highlighting}[]
\NormalTok{major_per_season }\OperatorTok\StringTok{ }\KeywordTok{summarise}\NormalTok{(}\DataTypeTok{mean_major_hurricane=} \KeywordTok{mean}\NormalTok{(NAMED_MAJOR_HURRICANES, }\DataTypeTok{na.rm=}\OtherTok{TRUE}\NormalTok{))}
\end{Highlighting}
\end{Shaded}

\begin{verbatim}
## # A tibble: 1 x 1
##   mean_major_hurricane
##                  <dbl>
## 1                 2.92
\end{verbatim}

This value is very close to 3 and I would consider the claim of 3 major
hurricanes to be true.

\begin{Shaded}
\begin{Highlighting}[]
\KeywordTok{ggplot}\NormalTok{(}\DataTypeTok{data=}\NormalTok{major_per_season, }\KeywordTok{aes}\NormalTok{(}\DataTypeTok{x=}\KeywordTok{factor}\NormalTok{(SEASON), }\DataTypeTok{y=}\NormalTok{NAMED_MAJOR_HURRICANES)) }\OperatorTok{+}
\KeywordTok{geom_bar}\NormalTok{(}\DataTypeTok{stat=}\StringTok{"identity"}\NormalTok{, }\DataTypeTok{color=} \StringTok{"black"}\NormalTok{, }\DataTypeTok{fill=} \StringTok{"lightblue"}\NormalTok{) }\OperatorTok{+}\StringTok{ }
\StringTok{  }\KeywordTok{scale_x_discrete}\NormalTok{(}\StringTok{"Year"}\NormalTok{, }\DataTypeTok{labels =} \DecValTok{1980}\OperatorTok{:}\DecValTok{2019}\NormalTok{, }\DataTypeTok{limits=} \KeywordTok{factor}\NormalTok{(}\DecValTok{1980}\OperatorTok{:}\DecValTok{2019}\NormalTok{)) }\OperatorTok{+}\StringTok{ }
\StringTok{  }\KeywordTok{labs}\NormalTok{(}\DataTypeTok{x=} \StringTok{"Year"}\NormalTok{, }\DataTypeTok{y=} \StringTok{"Number of Major Hurricanes"}\NormalTok{, }\DataTypeTok{title=} \StringTok{"Number of Named Category 3-5 Hurricanes per Year (1980-2019)"}\NormalTok{)}\OperatorTok{+}\StringTok{  }
\StringTok{  }\KeywordTok{geom_abline}\NormalTok{(}\DataTypeTok{slope=}\DecValTok{0}\NormalTok{, }\DataTypeTok{intercept=}\DecValTok{3}\NormalTok{,  }\DataTypeTok{col =} \StringTok{"black"}\NormalTok{, }\DataTypeTok{lwd=}\DecValTok{2}\NormalTok{) }\OperatorTok{+}
\StringTok{  }\KeywordTok{theme}\NormalTok{(}\DataTypeTok{axis.text.x =} \KeywordTok{element_text}\NormalTok{(}\DataTypeTok{angle =} \DecValTok{90}\NormalTok{)) }
\end{Highlighting}
\end{Shaded}

\includegraphics{KNFinalProject1_files/figure-latex/unnamed-chunk-34-1.pdf}

\hypertarget{final-conclusion-on-claim-b}{%
\subsubsection{Final Conclusion on Claim
B}\label{final-conclusion-on-claim-b}}

I believe there are about 12 storms per year. I do not believe there are
6 hurrianes, I believe there are closer to 7 hurricanes per year. I do
believe that there are about 3 major hurricanes per year.

\hypertarget{claim-c-true}{%
\section{Claim C: True}\label{claim-c-true}}

\hypertarget{c-september-is-the-most-active-month-where-most-of-the-hurricanes-occur-followed-by-august-and-october.}{%
\subsection{C) September is the most active month (where most of the
hurricanes occur), followed by August, and
October.}\label{c-september-is-the-most-active-month-where-most-of-the-hurricanes-occur-followed-by-august-and-october.}}

This question is similar to how I solved A; however, it specifically
asks for hurricanes so I will filter for only hurricane = true.

\begin{Shaded}
\begin{Highlighting}[]
\NormalTok{hurricanedat <-}\StringTok{ }\KeywordTok{filter}\NormalTok{(dat3, HURRICANE }\OperatorTok{==}\StringTok{ "TRUE"}\NormalTok{)}
\KeywordTok{head}\NormalTok{(hurricanedat,}\DecValTok{5}\NormalTok{)}
\end{Highlighting}
\end{Shaded}

\begin{verbatim}
##             SID SEASON NUMBER BASIN SUBBASIN  NAME            ISO_TIME NATURE
## 1 1980214N11330   1980     57    NA       NA ALLEN 1980-08-03 00:00:00     TS
## 2 1980214N11330   1980     57    NA       NA ALLEN 1980-08-03 06:00:00     TS
## 3 1980214N11330   1980     57    NA       NA ALLEN 1980-08-03 12:00:00     TS
## 4 1980214N11330   1980     57    NA       NA ALLEN 1980-08-03 18:00:00     TS
## 5 1980214N11330   1980     57    NA       NA ALLEN 1980-08-04 00:00:00     TS
##    LAT   LON WMO_WIND WMO_PRES WMO_AGENCY TRACK_TYPE DIST2LAND LANDFALL MONTH
## 1 12.4 -51.4       65      985 hurdat_atl       main       790      762     8
## 2 12.6 -53.6       70      980 hurdat_atl       main       753      713     8
## 3 12.8 -55.6       80      975 hurdat_atl       main       633      543     8
## 4 12.9 -57.5       95      965 hurdat_atl       main       453      398     8
## 5 13.3 -59.1      110      950 hurdat_atl       main       357      321     8
##   HURRICANE CATEGORY
## 1      TRUE        1
## 2      TRUE        1
## 3      TRUE        1
## 4      TRUE        2
## 5      TRUE        3
\end{verbatim}

\begin{Shaded}
\begin{Highlighting}[]
\NormalTok{hurricane_month <-}\StringTok{ }\KeywordTok{distinct}\NormalTok{(}\KeywordTok{select}\NormalTok{(hurricanedat, SID, SEASON, MONTH))}
\KeywordTok{head}\NormalTok{(hurricane_month,}\DecValTok{5}\NormalTok{)}
\end{Highlighting}
\end{Shaded}

\begin{verbatim}
##             SID SEASON MONTH
## 1 1980214N11330   1980     8
## 2 1980227N13325   1980     8
## 3 1980234N36287   1980     8
## 4 1980245N16322   1980     9
## 5 1980249N18336   1980     9
\end{verbatim}

Next I will filter for just the months in question: August, September,
and October

\begin{Shaded}
\begin{Highlighting}[]
\NormalTok{fall_months <-}\StringTok{ }\KeywordTok{filter}\NormalTok{(hurricane_month, MONTH }\OperatorTok\StringTok{ }\DecValTok{8}\OperatorTok{:}\DecValTok{10}\NormalTok{)}
\KeywordTok{head}\NormalTok{(fall_months,}\DecValTok{5}\NormalTok{)}
\end{Highlighting}
\end{Shaded}

\begin{verbatim}
##             SID SEASON MONTH
## 1 1980214N11330   1980     8
## 2 1980227N13325   1980     8
## 3 1980234N36287   1980     8
## 4 1980245N16322   1980     9
## 5 1980249N18336   1980     9
\end{verbatim}

Note: some of the same SID appear more than once because some of the
same storms occur in more than one month. For instance, SID:
1981265N14328 occurs in month 9 and 10 of 1981. I am going to count this
as a hurricane occuring in the month of 9 and a hurricane occuring in
the month of 10.

\begin{Shaded}
\begin{Highlighting}[]
\NormalTok{fall_hurricanes <-}\StringTok{ }\KeywordTok{summarise}\NormalTok{(}
  \KeywordTok{group_by}\NormalTok{(fall_months, SEASON, MONTH),}
  \DataTypeTok{number =} \KeywordTok{n}\NormalTok{()}
\NormalTok{)}
\end{Highlighting}
\end{Shaded}

\begin{verbatim}
## `summarise()` regrouping output by 'SEASON' (override with `.groups` argument)
\end{verbatim}

\begin{Shaded}
\begin{Highlighting}[]
\KeywordTok{head}\NormalTok{(fall_hurricanes,}\DecValTok{5}\NormalTok{)}
\end{Highlighting}
\end{Shaded}

\begin{verbatim}
## # A tibble: 5 x 3
## # Groups:   SEASON [2]
##   SEASON MONTH number
##    <int> <dbl>  <int>
## 1   1980     8      3
## 2   1980     9      3
## 3   1980    10      1
## 4   1981     8      1
## 5   1981     9      5
\end{verbatim}

\hypertarget{boxplot-of-fall-months}{%
\subsection{Boxplot of Fall Months}\label{boxplot-of-fall-months}}

I used a box plot to see the fall months' means and if there are any
outliers skewing the means.

\begin{Shaded}
\begin{Highlighting}[]
\KeywordTok{ggplot}\NormalTok{(fall_hurricanes, }\KeywordTok{aes}\NormalTok{(}\DataTypeTok{x=} \KeywordTok{factor}\NormalTok{(MONTH), }\DataTypeTok{y=}\NormalTok{number)) }\OperatorTok{+}
\StringTok{  }\KeywordTok{geom_boxplot}\NormalTok{(}\DataTypeTok{color=} \StringTok{"black"}\NormalTok{, }\DataTypeTok{fill=} \StringTok{"lightblue"}\NormalTok{, }\DataTypeTok{width=} \FloatTok{0.9}\NormalTok{) }\OperatorTok{+}\StringTok{ }\KeywordTok{labs}\NormalTok{(}\DataTypeTok{x=} \StringTok{"Month"}\NormalTok{, }\DataTypeTok{y=}\StringTok{"Frequency"}\NormalTok{, }\DataTypeTok{title=} \StringTok{"Number of Hurricanes that Occur Each Month Between 1980-2019"}\NormalTok{)}
\end{Highlighting}
\end{Shaded}

\includegraphics{KNFinalProject1_files/figure-latex/unnamed-chunk-39-1.pdf}

This boxplot shows that September definitely has the highest mean and
therefore the most amount of hurricane occurances. September is a clear
winner for most active month. From this boxplot it is difficult to tell
a difference between August and October.

I calculated the means to give me a better idea since I can't
differentiate the means on the boxplot.

\begin{Shaded}
\begin{Highlighting}[]
\NormalTok{fall_avg <-}\StringTok{ }\KeywordTok{summarise}\NormalTok{(}
  \KeywordTok{group_by}\NormalTok{(fall_hurricanes, MONTH),}
  \DataTypeTok{month_avg=} \KeywordTok{mean}\NormalTok{(number, }\DataTypeTok{na.rm=}\OtherTok{TRUE}\NormalTok{))}
\end{Highlighting}
\end{Shaded}

\begin{verbatim}
## `summarise()` ungrouping output (override with `.groups` argument)
\end{verbatim}

\begin{Shaded}
\begin{Highlighting}[]
\NormalTok{fall_avg}
\end{Highlighting}
\end{Shaded}

\begin{verbatim}
## # A tibble: 3 x 2
##   MONTH month_avg
##   <dbl>     <dbl>
## 1     8      2.09
## 2     9      3.15
## 3    10      1.97
\end{verbatim}

The means confirm that September is the most active month, followed by
August, then October as the claim states. I also want to visualize the
differences between August and October to get a closer look.

\begin{Shaded}
\begin{Highlighting}[]
\NormalTok{AugOct <-}\StringTok{ }\KeywordTok{filter}\NormalTok{(fall_hurricanes, MONTH }\OperatorTok\StringTok{ }\DecValTok{8} \OperatorTok{|}\StringTok{ }\NormalTok{MONTH }\OperatorTok\StringTok{ }\DecValTok{10}\NormalTok{)}
\KeywordTok{head}\NormalTok{(AugOct, }\DecValTok{5}\NormalTok{)}
\end{Highlighting}
\end{Shaded}

\begin{verbatim}
## # A tibble: 5 x 3
## # Groups:   SEASON [3]
##   SEASON MONTH number
##    <int> <dbl>  <int>
## 1   1980     8      3
## 2   1980    10      1
## 3   1981     8      1
## 4   1981    10      1
## 5   1983     8      2
\end{verbatim}

\begin{Shaded}
\begin{Highlighting}[]
\KeywordTok{ggplot}\NormalTok{(AugOct, }\KeywordTok{aes}\NormalTok{(}\DataTypeTok{x=}\NormalTok{ SEASON, }\DataTypeTok{y=}\NormalTok{number, }\DataTypeTok{fill=} \KeywordTok{factor}\NormalTok{(MONTH))) }\OperatorTok{+}\StringTok{ }
\StringTok{  }\KeywordTok{geom_bar}\NormalTok{(}\DataTypeTok{stat =} \StringTok{"identity"}\NormalTok{)}\OperatorTok{+}
\StringTok{  }\KeywordTok{labs}\NormalTok{(}\DataTypeTok{x=} \StringTok{"Year"}\NormalTok{, }\DataTypeTok{y=}\StringTok{"Frequency"}\NormalTok{, }\DataTypeTok{title=} \StringTok{"Number of Hurricanes for August and October Between 1980-2019"}\NormalTok{)}
\end{Highlighting}
\end{Shaded}

\includegraphics{KNFinalProject1_files/figure-latex/unnamed-chunk-42-1.pdf}

From looking at this graph, it is clear to see that August surpasses
October every year.

\hypertarget{final-conclusion-on-claim-c}{%
\subsubsection{Final Conclusion on Claim
C}\label{final-conclusion-on-claim-c}}

Given all the evidence, I definitely agree Claim C is true.

\hypertarget{claim-d-true}{%
\section{Claim D: True}\label{claim-d-true}}

\hypertarget{d-during-the-analyzed-period-1980-2019-no-hurricanes-made-u.s.-landfall-before-june-and-after-november.}{%
\subsection{D) During the analyzed period (1980-2019), no hurricanes
made U.S. landfall before June and after
November.}\label{d-during-the-analyzed-period-1980-2019-no-hurricanes-made-u.s.-landfall-before-june-and-after-november.}}

To assess this claim, I first filtered the data frame to only include
hurricanes and only the time frame specified.

\begin{Shaded}
\begin{Highlighting}[]
\NormalTok{hurricanetime <-}\StringTok{ }\KeywordTok{filter}\NormalTok{(dat3, HURRICANE }\OperatorTok{==}\StringTok{ "TRUE"}\NormalTok{, MONTH }\OperatorTok{<}\DecValTok{6} \OperatorTok{|}\StringTok{ }\NormalTok{MONTH }\OperatorTok{>}\DecValTok{11}\NormalTok{)}
\KeywordTok{head}\NormalTok{(hurricanetime,}\DecValTok{5}\NormalTok{)}
\end{Highlighting}
\end{Shaded}

\begin{verbatim}
##             SID SEASON NUMBER BASIN SUBBASIN NAME            ISO_TIME NATURE
## 1 1984348N35300   1984    121    NA       NA LILI 1984-12-20 12:00:00     TS
## 2 1984348N35300   1984    121    NA       NA LILI 1984-12-20 18:00:00     TS
## 3 1984348N35300   1984    121    NA       NA LILI 1984-12-21 00:00:00     TS
## 4 1984348N35300   1984    121    NA       NA LILI 1984-12-21 06:00:00     TS
## 5 1984348N35300   1984    121    NA       NA LILI 1984-12-21 12:00:00     TS
##    LAT   LON WMO_WIND WMO_PRES WMO_AGENCY TRACK_TYPE DIST2LAND LANDFALL MONTH
## 1 31.1 -52.4       70      980 hurdat_atl       main      1726     1726    12
## 2 30.5 -52.3       70      980 hurdat_atl       main      1794     1794    12
## 3 30.0 -52.2       70      980 hurdat_atl       main      1850     1850    12
## 4 29.5 -52.1       70      980 hurdat_atl       main      1846     1831    12
## 5 29.0 -52.0       70      980 hurdat_atl       main      1818     1805    12
##   HURRICANE CATEGORY
## 1      TRUE        1
## 2      TRUE        1
## 3      TRUE        1
## 4      TRUE        1
## 5      TRUE        1
\end{verbatim}

I then was curious just to see what these hurricanes looked like on a
map. (This doesn't answer the question, just provides a good overall
look)

\begin{Shaded}
\begin{Highlighting}[]
\NormalTok{gg_world }\OperatorTok{+}\StringTok{ }
\StringTok{  }\KeywordTok{geom_point}\NormalTok{(}\DataTypeTok{data =}\NormalTok{ hurricanetime, }\KeywordTok{aes}\NormalTok{(}\DataTypeTok{x =}\NormalTok{ LON, }\DataTypeTok{y =}\NormalTok{ LAT, }\DataTypeTok{color =}\NormalTok{ NAME))}
\end{Highlighting}
\end{Shaded}

\includegraphics{KNFinalProject1_files/figure-latex/unnamed-chunk-44-1.pdf}

This map shows me that only one hurricane, Andrea, seems relatively
close to the U.S. This map also shows me that one hurricane, Barbra,
appears to have made landfall.

\hypertarget{filtering-landfall}{%
\paragraph{Filtering landfall}\label{filtering-landfall}}

The data dictionary specifies a hurricane to make landfall when it is
=0. The data dictionary specifies that values less than ``60 nmile'' are
likely to impact the land. I am going to assume that they meant to say
60 miles. The landfall category is recorded in km so I will convert 60
miles to km which is 96.5. I am going to filter for hurricanes that have
a landfall below 97 because I want to see which hurricanes have an
impact on the land.

\begin{Shaded}
\begin{Highlighting}[]
\NormalTok{Barb <-}\StringTok{ }\KeywordTok{filter}\NormalTok{(hurricanetime, LANDFALL }\OperatorTok{<}\StringTok{ }\DecValTok{97}\NormalTok{)}
\NormalTok{Barb}
\end{Highlighting}
\end{Shaded}

\begin{verbatim}
##             SID SEASON NUMBER BASIN SUBBASIN    NAME            ISO_TIME NATURE
## 1 2013149N14264   2013     22    EP     <NA> BARBARA 2013-05-29 18:00:00     TS
## 2 2013149N14264   2013     22    EP     <NA> BARBARA 2013-05-29 19:50:00     TS
##    LAT   LON WMO_WIND WMO_PRES WMO_AGENCY TRACK_TYPE DIST2LAND LANDFALL MONTH
## 1 15.7 -94.2       65      986 hurdat_epa       main        39        0     5
## 2 16.0 -94.0       70      983 hurdat_epa       main         0        0     5
##   HURRICANE CATEGORY
## 1      TRUE        1
## 2      TRUE        1
\end{verbatim}

Only one hurricane resulted in this filter and it was Barbara who's
landfall is = 0.

\hypertarget{visualization-of-barbara}{%
\paragraph{Visualization of Barbara}\label{visualization-of-barbara}}

\begin{Shaded}
\begin{Highlighting}[]
\NormalTok{gg_world }\OperatorTok{+}\StringTok{ }
\StringTok{  }\KeywordTok{geom_point}\NormalTok{(}\DataTypeTok{data =}\NormalTok{ Barb, }\KeywordTok{aes}\NormalTok{(}\DataTypeTok{x =}\NormalTok{ LON, }\DataTypeTok{y =}\NormalTok{ LAT, }\DataTypeTok{color =}\NormalTok{ NAME))}
\end{Highlighting}
\end{Shaded}

\includegraphics{KNFinalProject1_files/figure-latex/unnamed-chunk-46-1.pdf}

\hypertarget{final-conclusion-on-claim-d}{%
\subsubsection{Final Conclusion on Claim
D}\label{final-conclusion-on-claim-d}}

Barbara is not located in the U.S. so we can safely say that no
hurricanes between the time period 1980-2019 before the months June and
after the months November made landfall.

\hypertarget{thank-for-reading-my-report}{%
\subsubsection{Thank for reading my report
!}\label{thank-for-reading-my-report}}

\end{document}
